%Dokumentenklasse "scrbook" - Erweitert um den Verweis auf die Verzeichnisse und Texteigenschaften
\documentclass[chapterprefix=true, 12pt, a4paper, oneside, parskip=half, listof=totoc, bibliography=totoc, numbers=noendperiod]{scrbook}

% Ränder (Standard bottom ca. 52mm anbzüglich von ca. 4mm für die nach oben rechts gewanderte Seitenzahl)
%Anpassung der Seitenränder
\usepackage[bottom=32mm,left=25mm,right=25mm]{geometry}

% Ränder bei Bedarf zeigen
%\usepackage{showframe}

%Tweaks für scrbook
\usepackage{scrhack}



%Blindtext
\usepackage{blindtext}

%Erlaubt unteranderem Umbrücke captions
\usepackage{caption}

%Stichwortverzeichnis
\usepackage{imakeidx}

%Kompakte Listen
\usepackage{paralist}

%Zitate besser formatieren und darstellen
\usepackage{epigraph}

%Glossar + Stichworverzeichnis
\usepackage[toc, acronym]{glossaries} % Akronyme werden als eigene Liste aufgeführt
\glsenablehyper

%Anpassung von Kopf- und Fußzeile
%beinflusst die erste Seite des Kapitels
\usepackage[automark,headsepline]{scrlayer-scrpage}
\automark{chapter}
\ihead{\leftmark}
\chead{}
\ohead{\thepage}
\ifoot*{}
\cfoot[\thepage]{}
\cfoot*{}
\ofoot*{}
\pagestyle{scrheadings}

%Auskommentieren für die Verkleinerung des vertikalen Abstandes eines neuen Kapitels
%\renewcommand*{\chapterheadstartvskip}{\vspace*{.25\baselineskip}}

%Zeilenabstand 1,5
\usepackage[onehalfspacing]{setspace}

%Verbesserte Darstellung der Buchstaben zueinander
\usepackage[stretch=10]{microtype}

%Deutsche Bezeichnungen für angezeigte Namen (z.B. Innhaltsverzeichnis etc.)
%\usepackage[ngerman]{babel}
\usepackage{babel}
\babelprovide[import]{english}
\babelprovide[main, import]{ngerman}



%Unterstützung von Umlauten und anderen Sonderzeichen (UTF-8)
\usepackage{lmodern}
\usepackage[utf8]{luainputenc}
\usepackage[T1]{fontenc}

%Einfachere Zitate
\usepackage{epigraph}

%Verwendung von Akronymen
\usepackage[printonlyused]{acronym}

%Unterstützung der H positionierung (keine automatische Verschiebung eingefügter Elemente)
\usepackage{float} 

%Erlaubt Umbrüche innerhalb von Tabellen
\usepackage{tabularx}

%Erlaubt Seitenumbrüche mit Tabellen
\usepackage{longtable}

%Erlaubt die Darstellung von Sourcecode mit Highlighting
\usepackage{listings}

%Definierung eigener Farben bei nutzung eines selbst vergebene Namens
\usepackage[table,xcdraw]{xcolor}

%Vektorgrafiken tikz
\usepackage{tikz}

%Grafiken (wie jpg, png, etc.)
\usepackage{graphicx}

%Grafiken von Text umlaufen lassen
\usepackage{wrapfig}

%Grafik bestehend aus mehreren Grafiken
\usepackage{subcaption}

%Ermöglicht Verknüpfungen innerhalb des Dokumentes (e.g. for PDF), Links werden durch "hidelink" nicht explizit hervorgehoben
\usepackage[hidelinks,ngerman]{hyperref}

%Einbindung und Verwaltung von Literaturverzeichnissen
\usepackage{csquotes} %wird von biber benötigt
\usepackage[style=numeric, backend=biber, bibencoding=utf8,defernumbers=true]{biblatex}
\addbibresource{references/references.bib}
%enable the macro \mathbb
\usepackage{amssymb}
\usepackage{amsmath}

%-------------------------------Zusätzliche Anpassungen und Modifikationen--------------------------------------------%

%Anpassung der Überschriften
\addtokomafont{disposition}{\rmfamily}

%Zusätzliche Farben
\definecolor{Ao}{rgb}{0.0, 0.39, 0.0}
\definecolor{antiquefuchsia}{rgb}{0.57, 0.36, 0.51}
\definecolor{bostonuniversityred}{rgb}{0.8, 0.0, 0.0}

% Padding für images innerhalb von longtables
\usepackage{verbatimbox}
\newcommand\Includegraphics[2][]{\addvbuffer[5pt 0pt]{\includegraphics[#1]{#2}}}

%Umbenennungen
\renewcommand{\lstlistlistingname}{Source Code Content}

%Pluszeichen in der Reference beim zitieren ausblenden
\renewcommand*{\labelalphaothers}{}

%Anpassugen zur Quelltextdarstellung, kann bei Bedarf überschrieben werden (z.B. wenn unterschiedliche Sprachen zum Einsatz kommen)
\renewcommand{\lstlistingname}{Code snippet}
\lstset{
	language=python,
	numbers=left,
	columns=fullflexible,
	aboveskip=5pt,
	belowskip=10pt,
	basicstyle=\small\ttfamily,
	backgroundcolor=\color{black!5},
	commentstyle=\color{black},
	morecomment=[s]{.s}{hape}, % workaround to exclude ".shape"
	keywordstyle=\color{antiquefuchsia},
	otherkeywords={self, else},
	emphstyle=\color{bostonuniversityred},
	emph={shape,units,name,filters,kernel_size,activation,padding,kernel_initializer,kernel_regularizer,bias_initializer,inputs,outputs,mode,distribution,minval,maxval, callback_args, low, high, limit, window_length, size, mu, theta, sigma},
	stringstyle=\color{Ao},
	showspaces=false,
	showstringspaces=false,
	showtabs=false,
	xleftmargin=16pt,
	xrightmargin=0pt,
	framesep=5pt,
	framerule=1pt,
	frame=leftline,
	rulecolor=\color{black},
	tabsize=2,
	breaklines=true,
	breakatwhitespace=true
}

%Anpassungen für das Abkürzungsverzeichnis
\newglossarystyle{dottedlocations}{%
	\glossarystyle{list}%
	\renewcommand*{\glossaryentryfield}[5]{%
		\item[\glsentryitem{##1}\glstarget{##1}{##2}] \emph{##3}%
		\unskip\leaders\hbox to 2.9mm{\hss.}\hfill##5}%
	\renewcommand*{\glsgroupskip}{}%
}